\section{Introduction}
\label{sec:introduction}

Grand challenges of cyber-physical systems (CPS) include a high
computational cost of understanding the physical world.
Object detection is one of compute-intensive tasks for CPS.
For example, an autonomous vehicle needs to detect and track other
vehicles by itself.
Current autonomous driving technologies \cite{Guizzo11, Levinson11,
Urmson08} tend to rely on active sensors such as GPS, RADAR, and LIDAR
\cite{Kirchner00, Streller02} together with very accurate pre-configured
maps, but the use of passive camera sensors is becoming more practical
due to recent advances in computer vision \cite{Dalal05, Felzenszwalb05,
Felzenszwalb10}: vision-based object detection can be applied for
various ranges and orientations.
In particular, histograms of oriented gradients (HOG) \cite{Dalal05}
features provide reliable high-level representations of an image
underlying many state-of-the-art object detection
algorithms~\cite{Felzenszwalb10, Geiger12, Rybski10, Suard06, Zhu06}.
However, a major concern of HOG-based object detection remains in
computational cost.

Previous work on the implementation of HOG-based object detection are
limited to either hardware implementations \cite{Kadota09, Karakaya09,
Komorkiewicz12} or specific parts of HOG algorithms \cite{Chen11,
Prisacariu09}.
There is even no quantitative investigation of what implementation
issues could prevent HOG-based object detection from being deployed in
real-world applications.
Given recent innovations in commodity hardware technology such as
multicores and graphics processing units (GPUs), it is worth
exploring if the current state of the arts can meet computational
requirements of cutting-edge object detection implementations.

\textbf{Contribution:}
This paper presents GPU implementations of HOG-based object detection in
consideration of real-world applications using deformable part models
\cite{Felzenszwalb10}.
While this is a popular vision-based object detection approach, what
remains an open question is a generalized programming technique and a
quantification of performance characteristics for practical use.
We begin with an analysis of traditional CPU
implementations to find fundamental performance bottlenecks of HOG-based
object detection.
This analysis reasons about our approach to GPU implementations that we
offload only the detected compute-intensive blocks of the object
detection program to the GPU.
The experimental results obtained from a real-world vehicle detection
program show that commodity GPUs outperform high-performance multi-core
CPUs by 3x to 5x in frame-rate, though at least another 5x improvement
would be needed to deploy in the real world.

\textbf{Organization:}
The rest of this paper is organized as follows.
Section~\ref{sec:assumption} describes the assumption behind this paper.
Section~\ref{sec:implementation} presents an analysis of HOG-based
object detection and our GPU implementation technique.
Section~\ref{sec:evaluation} evaluates the performance benefit of our
technique over traditional CPU implementations.
This paper concludes in Section~\ref{sec:conclusion}.