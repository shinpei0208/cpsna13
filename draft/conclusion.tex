\section{Conclusion}
\label{sec:conclusion}

In this paper, we have presented GPU implementations of HOG-based object
detection and their performance evaluation.
Unlike preceding work that highly stressed on performance improvements,
our implementations are based on an analysis of performance bottlenecks
posed due to an introduction of the deformable models in HOG-based
object detection.
This approach ensures that the GPU truly accelerates approapriate
computational blocks.
Our evaluation using a commodity GPU showed that our GPU implementation
can speed up the existing HOG-based vehicle detection program tailored
to the deformable models by 1.5x to 3x over traditional CPU
implementations.
Given that this performance improvement is obtained from the entire
program runtime rather than particular algorithm parts of the program,
our contribution is useful and significant for real-world applications
of vision-based object detection. 

To the best of our knowledge, this is the first piece of work that made
a \textit{tight} coordination of object detection and parallel computing
-- a core challenge of CPS.
Specifically we showed that a measured and structured way of GPU
programming is efficient for the object detection program and quantified
the impact of GPUs in performance.
Our conclusion is that GPUs are promising to meet the required
performance of vision-based object detection in the real world.

In future work, we plan to complement this work with systemized
coordinations of computations and I/O devices.
Since real-world applications require camera sensors to obtain input
images while GPUs are compute devices off the host computer, the data
I/O latency could become a non-trivial bottleneck on the data bus.
In this scenario, we need enhanced system support such as zero-copy
approaches \cite{Kato13} to minimize the data latency raised between
camera sensors and GPUs.
We also plan to augment our GPU implementations using multiple GPUs in
order to meet the real-time and real-fast requirement of real-world CPS
applications.
